\setalias{\map}{\texttt{map\_domain\_page()}\xspace}
\setalias{\unmap}{\texttt{unmap\_domain\_page()}\xspace}
\setalias{\ulong}{\texttt{unsigned long}\xspace}
\setalias{\voidp}{\texttt{void *}\xspace}
\setalias{\vcache}{\texttt{mapcache\_vcpu}\xspace}
\setalias{\dcache}{\texttt{mapcache\_domain}\xspace}
\setalias{\hashent}{\texttt{vcpu\_maphash\_entry}\xspace}
\setalias{\inuse}{\texttt{inuse}\xspace}
\setalias{\garbage}{\texttt{garbage}\xspace}

\paragraph{}
The map cache is the system designed to allow you accessing any page frame
content.
Because even in hypervisor mode, the current processor still use the domain
specific page table, the map cache has to create a mapping between the page
frame you want to access to and a virtual address.

\paragraph{}
This system is based on the two main functions: \map and \unmap.
Their respectively take an \ulong, which represent the machine frame number,
map it and give you back a \voidp, which represent the physical accessible
address ; or take a previously returned \voidp and unmap it.
It is possible to map the same mfn many times, a reference count is keeped
inside the map cache, so it will stay mapped at the same physical address until
it has been unmapped as many times it has been mapped\footnote{see below the
limitations of the map cache reference counting}.

\paragraph{}
With this two core functions, there also is \texttt{clear\_domain\_page()}
which clears the content of a given machine frame number and
\texttt{copy\_domain\_page()} which copies the content of its second \ulong
machine frame number argument into its first one.
These two functions automatically map and unmap the given frame numbers.
Another function \texttt{domain\_page\_map\_to\_mfn()} can be usefull to
retrieve the machine frame number from a given physical address.

\paragraph{}
The mapping created by \map and \unmap are local to the current vcpu.
More precisely, the physical addresses returned by \map can only be cached for
the current vcpu, meaning if two vcpus map the same machine frame number many
times, there will be two valid mappings, usable by every vcpus of the domain.
Most of the time, it is what you want because the operations on machine frames
are often local and this way avoid some unecessary sharing between vcpus.
But sometimes, you want to map a machine frame number into a physical address
intended to be used by several vcpus.
That is the goal of the \texttt{map\_domain\_page\_global()} and
\texttt{unmap\_domain\_page\_global()} functions which do the same thing than
their local sisters but the physical addresses do not go throught the local
cache system, instead their use another global mapping mechanism.

\paragraph{}
Now we have seen the interface provided by the map cache system, let us see the
internals and the limitations of it.
The main structures are described in the \texttt{xen/include/asm-x86/domain.h}
header file and are named \dcache, \vcache and \hashent.
In the first time, we will only see how the local mappings are performed.
The global mapping will be described briefly because it is based on other
mechanisms of xen.

\paragraph{}
The mapping is performed using the linear mapping tables which are tables of
page table entries, belonging to the domain page table\footnote{I have not
actually seen where, but it is very likely to be so}, and mapped on physical
addresses, so their are directly accessible.
That means changes written in these tables are directly seen by the MMU (if
the TLB flushes are performed correctly).
The structure \dcache is the one responsible to maintain the information of
what is mapped where.
It is mainly composed of two bitsets, each bit of these correspond to an entry
in the linear tables.
The first one is \inuse and tracks if a machine frame has been mapped on a
given entry since the last TLB flush.
The second one is \garbage and tracks what entry has been unmapped on the
linear table but not yet flushed.
The goal to have these to bitsets is to batch the TLB flushes since it is a
very costly operation.

\paragraph{}
In addition of these bitsets, come the fields \texttt{entries} and
\texttt{cursor} which indicate respectively the capacity of the bitsets (it is
a constant field), and what is the next entry to search on to find a free slot.
Please notice it does not mean the entrie is actually free, it is just a way
to check the entries in a circular way instead of checking from the entry 0
each time.
The \dcache structure is depicted in figure \ref{figure dcache scheme}.

\begin{figure}
  \centering
  \begin{tikzpicture}[start chain=1 going above, start chain=2 going right,
      start chain=3 going right, start chain=4 going above, node distance=0]
    \tikzstyle{dcache}=[draw,on chain=1, minimum width=2.3cm,
      minimum height=.8cm]
    \tikzstyle{bitset}=[draw,minimum width=.45cm,minimum height=.5cm]
    \tikzstyle{inuse}=[bitset,on chain=2]
    \tikzstyle{garbage}=[bitset,on chain=3]
    \tikzstyle{linear}=[draw,on chain=4,minimum width=1.6cm,
      minimum height=.4cm]
    \def\z{\texttt{0}}
    \def\o{\texttt{1}}

    \node[dcache,anchor=south east] (garbage) at (0,.2) {\garbage};
    \node[dcache] (inuse) {\inuse};
    \node[dcache] (cursor) {\texttt{cursor}};
    \node[dcache] (entries) {\texttt{entries}};
    \node[anchor=north] at ($(garbage.south)+(0,-.3)$) {\dcache};

    \node[garbage,anchor=south west] (garbage0) at (1,0) {\o};
    \foreach \i in {1,2,3} \node[garbage] (garbage\i) {\z};
    \foreach \i in {4,5,6,7} \node[garbage] (garbage\i) {\z};

    \node[inuse,anchor=south west] (inuse0) at (1,1) {\o};
    \foreach \i in {1,2,3} \node[inuse,fill=gray!30] (inuse\i) {\o};
    \node[inuse] (inuse4) {\z};
    \node[inuse] (inuse5) {\z};
    \node[inuse,fill=gray!30] (inuse6) {\o};
    \node[inuse] (inuse7) {\z};

    \node[on chain=4,minimum height=1cm] at (7,-.6)
         {\texttt{\_\_linear\_l1\_table}};
    \node[linear] (linear7) {};
    \node[linear,fill=gray!30] (linear6) {};
    \foreach \i in {5,4} \node[linear] (linear\i) {};
    \foreach \i in {3,2,1} \node[linear,fill=gray!30] (linear\i) {};
    \node[linear] (linear0) {};

    \draw[->] (inuse.east) -- (inuse0.west);
    \draw[->] (garbage.east) -- (garbage0.west);

    \draw[<->] ($(garbage5.south west)+(0,-.3)$) -- node[below] {bit}
    ($(garbage5.south east)+(0,-.3)$);
    \node (entrieswest) at ($(inuse0.north west)+(0,1)$) {};
    \node (entrieseast) at ($(inuse7.north east)+(0,1)$) {};
    \draw[<->] (entrieswest.center) -- node (entriescenter) {}
    (entrieseast.center);
    \draw[dashed,color=gray] (inuse0.north west) -- (entrieswest.center);
    \draw[dashed,color=gray] (inuse7.north east) -- (entrieseast.center);
    \draw[<->] ($(linear0.north west)+(-.2,0)$) --
    node[left] {\texttt{l1\_pgentry\_t}} ($(linear0.south west)+(-.2,0)$);

    \draw[->,rounded corners] (cursor.east) -| (inuse4.north);
    \draw[rounded corners] (entries.east) -| (entriescenter.center);

  \end{tikzpicture}
  \caption{\label{figure dcache scheme}Representation of the \dcache with
    the \inuse and \garbage bitsets.
    Beside the linear mapping \texttt{\_\_linear\_l1\_table} with mapping slots
    in gray.
    These slots match the entries which are \texttt{1} in the \inuse bitset and
    \texttt{0} in the \garbage one.}
\end{figure}

\paragraph{}
When the \map function is called to map a machine page, the map cache system:
\begin{itemize}
\item searches, starting from the cursor, an entry in the \inuse bitset which
  is \texttt{0}
\item if it reaches the end of the bitset without finding, it will trigger a
  ``garbage collection'', meaning all entries which have a \texttt{1}
  in the \garbage bitset will be set to \texttt{0} in both bitsets, then
  it reset the cursor and restart a search
\item if it does not find any entry which can be used after this
  ``garbage collection'', Xen crashes
\end{itemize}
If a collection occurs, the map cache system will also perform a TLB flush so
the MMU can know the garbage entries are not mapped anymore.
When the map cache system has an entry, it sets the matching \inuse bit to
\texttt{1} and set the cursor to the next entry.
Then it writes the machine frame number in the appropriate slot of the linear
table and return the physical address, which is simply a predefined physical
address incremented by as many pages as the entry number.
When the \unmap function is called, the map cache system starts by computing
the entry number, basing on the given physical address, then it set the
\garbage bit to \texttt{1}, meaning it can be flused for the next
``collection''.

\paragraph{}
Now, as described here, this system could work, but it is often the same
machine pages which need to be mapped (for instance, the page table pages
theirselves), and it would be more efficient to reuse these mappings with a
cache system.
It is the goal of the two other structures: \vcache and \hashent.
There is one \vcache structure per vcpu and a fixed amount of \hashent
(eight in the current implementation) per \vcache.
The state flow of an \hashent is depicted in \ref{figure hashent scheme}.

\paragraph{}
A \hashent structure is simply a container for three fields as described in
table \ref{table vcpu_maphash_entry}.
\begin{table}
  \centering
  \begin{tabularx}{\textwidth}{ | l | X | }
    \hline
    field   & description \\
    \hline
    \texttt{mfn}    & The machine frame number of the cached entry.
                      When the \hashent is not used, this field is set to
                      $\sim$\texttt{0}. \\
    \hline
    \texttt{idx}    & The physical entry of the cached page.
                      When the \hashent is not used, this field is set to
                      \texttt{MAPHASHENT\_NOTINUSE}. \\
    \hline
    \texttt{refcnt} & The amount of time this entry is currently mapped.
                      When the \hashent is not used, this field is set to
                      \texttt{0}. \\
    \hline
  \end{tabularx}
  \caption{\label{table vcpu_maphash_entry}Definition of
    \texttt{vcpu\_maphash\_entry} fields}
\end{table}
The \vcache is simply an array of \hashent.
The hash function is a binary mask of the machine frame number to keep only the
lowest bits.
The main thing to remember about the actual cache of the map cache system is it
work in the inverted way comparing to a hardware cache: an machine mapping can
be cached when it is unmapped by the user.
However, a cache entry can be evicted only when the user want to create a new
machine mapping.

\paragraph{}
The caching mechanisms take place when the \map and the \unmap functions are
called.
At \unmap, as said above, the map cache starts by computing the entry number,
basing on the given physical address.
Then it can read in the mapping table (the one used by the MMU) to find the
machine frame number.
It can apply the hash function on this last to find the corresponding cache
entry.
Now because this is a cache, the cache entry may already be used for another
machine frame, so the map cache need to compare the \texttt{mfn} field of the
entry with the previously found frame number.
If it match, the \texttt{refcnt} field is decremented.

\paragraph{}
If the entry caches another mapping which can be evicted (its \texttt{refcnt}
field is \texttt{0}), it is.
Because the map entry stores the physical entry (\texttt{idx} field) of the
replaced frame, the map cache can find the corresponding bit in \garbage and
set it to \texttt{1}.
It can also remove the mapping from the page table since the physical entry
indicates the physical address of the page.
Then the physical entry and machine frame fields of the cache entry are
replaced.
It is also possible the cache entry be empty, in this case the \texttt{idx} and
\texttt{mfn} fields are also updated to match the newly unmapped frame.

\paragraph{}
If no entry can be found in these ways, the currently unmapped frame cannot be
cached for the next time.
The corresponding bit in \garbage is then set to \texttt{1} and the mapping is
removed from the page table.

\paragraph{}
Now when the \map function is called, before to search a slot in \inuse, the
map cache use the machine frame number to compute the index of the cache entry.
As for the \unmap function, it then compare the \texttt{mfn} field to the
current frame number.
If their match, the \texttt{refcnt} field of the entry is incremented and the
function returns immediately the physical address, since the page is already
mapped.

\paragraph{}
If the entry does not match with the currently mapped frame, then the \map
function goes back to the normal path.
The map cache is then only checked when no entry can be found in the \inuse
bitset and the first ``collection'' failed to free any slot.
In this case, every cache entries are checked and the first one which can be
evicted is.
Because this operation takes place after the ``garbage collection'' but before
the, now unavoidable, TLB flush, there is no need to use the \garbage bitset
and the entry is just removed from the page table.

\paragraph{}
When a cache entry is evicted this way, its physical entry (\texttt{idx}) is
used for the newly mapped machine frame.
However, the entry is marked as not in use.
This is because all cache entries are checked for eviction, and not only the
one matching the machine frame number.
If such an entry was marked as used, with a \texttt{refcnt} to \texttt{1}, it
could never be decremented by the \unmap function.
If no cache entry can be found in this way, the \map function goes back to its
normal path and Xen crashes.

\begin{figure}
  \centering
  \begin{tikzpicture}
    \def\ns{2.2cm}
    \def\fs{1.3cm}
    \def\hd{2}
    \def\vd{1.2}

    \node[draw,text width=\ns,minimum width=\ns] (unusedA) {
      \makebox[\ns][c]{\bf unused}\\
      \makebox[\fs][l]{\texttt{mfn}}\texttt{= }$\sim$\texttt{0}\\
      \makebox[\fs][l]{\texttt{idx}}\texttt{= }$\emptyset$\\
      \makebox[\fs][l]{\texttt{refcnt}}\texttt{= 0}
    };

    \node[draw,text width=\ns,minimum width=\ns,anchor=west] (cachedA)
    at ($(unusedA.east)+(\hd,0)$) {
      \makebox[\ns][c]{\bf cached}\\
      \makebox[\fs][l]{\texttt{mfn}}\texttt{= }$\delta$\\
      \makebox[\fs][l]{\texttt{idx}}\texttt{= }$\Delta$\\
      \makebox[\fs][l]{\texttt{refcnt}}\texttt{= 0}
    };

    \node[draw,text width=\ns,minimum width=\ns,anchor=west] (usedA)
    at ($(cachedA.east)+(\hd,0)$) {
      \makebox[\ns][c]{\bf used}\\
      \makebox[\fs][l]{\texttt{mfn}}\texttt{= }$\delta$\\
      \makebox[\fs][l]{\texttt{idx}}\texttt{= }$\Delta$\\
      \makebox[\fs][l]{\texttt{refcnt}}\texttt{> 0}
    };

    \node[draw,text width=\ns,minimum width=\ns,anchor=north] (cachedB)
    at ($(cachedA.south)+(0,-\vd)$) {
      \makebox[\ns][c]{\bf cached}\\
      \makebox[\fs][l]{\texttt{mfn}}\texttt{= }$\omega$\\
      \makebox[\fs][l]{\texttt{idx}}\texttt{= }$\Omega$\\
      \makebox[\fs][l]{\texttt{refcnt}}\texttt{= 0}
    };

    \node[minimum width=\ns,minimum height=2cm,anchor=east] (unusedB) at
    ($(cachedB.west)+(-\hd,0)$) {\bf \Large{\ldots}};
    \node[minimum width=\ns,minimum height=2cm,anchor=west] (usedB) at
    ($(cachedB.east)+(\hd,0)$) {\bf \Large{\ldots}};

    \draw[->,>=stealth'] (unusedA) edge[bend left] node[above]
         {\texttt{unmap(}$\Delta$\texttt{)}} (cachedA);
    \draw[->,>=stealth'] (cachedA) edge[bend left] node[above]
         {\texttt{map(}$\epsilon$\texttt{)}} (unusedA);

    \draw[->,>=stealth'] (cachedA) edge[bend left] node[above]
         {\texttt{map(}$\delta$\texttt{)}} (usedA);
    \draw[->,>=stealth'] (usedA) edge[bend left] node[above]
         {\texttt{unmap(}$\Delta$\texttt{)}} (cachedA);

    \draw[->,>=stealth'] (cachedA) edge[bend right] node[left]
         {\texttt{unmap(}$\Omega$\texttt{)}} (cachedB);
    \draw[->,>=stealth'] (cachedB) edge[bend right] node[right]
         {\texttt{unmap(}$\Delta$\texttt{)}} (cachedA);

    \draw[->,>=stealth'] (unusedB) edge[bend left] (cachedB);
    \draw[->,>=stealth'] (cachedB) edge[bend left] (unusedB);

    \draw[->,>=stealth'] (cachedB) edge[bend left] (usedB);
    \draw[->,>=stealth'] (usedB) edge[bend left] (cachedB);

    \draw[->,>=stealth'] ($(unusedA.north)+(-1,.5)$) -- (unusedA);

  \end{tikzpicture}
  \caption{\label{figure hashent scheme}State flow for a single \hashent
    cache entry structure.
    The machine frame numbers are mapped here are with the H function
    with H$(\delta) =$ H$(\omega) \neq$ H$(\epsilon)$.}
\end{figure}
