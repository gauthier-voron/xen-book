\setalias{\map}{\texttt{map\_domain\_page()}\xspace}
\setalias{\unmap}{\texttt{unmap\_domain\_page()}\xspace}
\setalias{\ulong}{\texttt{unsigned long}\xspace}
\setalias{\voidp}{\texttt{void *}\xspace}
\setalias{\vcache}{\texttt{mapcache\_vcpu}\xspace}
\setalias{\dcache}{\texttt{mapcache\_domain}\xspace}
\setalias{\hashent}{\texttt{vcpu\_maphash\_entry}\xspace}
\setalias{\inuse}{\texttt{inuse}\xspace}
\setalias{\garbage}{\texttt{garbage}\xspace}

\paragraph{}
The map cache is the system designed to allow you accessing any page frame
content.
Because even in hypervisor mode, the current processor still use the domain
specific page table, the map cache has to create a mapping between the page
frame you want to access to and a virtual address.

\paragraph{}
This system is based on the two main functions: \map and \unmap.
Their respectively take an \ulong, which represent the machine frame number,
map it and give you back a \voidp, which represent the physical accessible
address ; or take a previously returned \voidp and unmap it.
It is possible to map the same mfn many times, a reference count is keeped
inside the map cache, so it will stay mapped at the same physical address until
it has been unmapped as many times it has been mapped\footnote{see below the
limitations of the map cache reference counting}.

\paragraph{}
With this two core functions, there also is \texttt{clear\_domain\_page()}
which clears the content of a given machine frame number and
\texttt{copy\_domain\_page()} which copies the content of its second \ulong
machine frame number argument into its first one.
These two functions automatically map and unmap the given frame numbers.
Another function \texttt{domain\_page\_map\_to\_mfn()} can be usefull to
retrieve the machine frame number from a given physical address.

\paragraph{}
The mapping created by \map and \unmap are local to the current vcpu.
More precisely, the physical addresses returned by \map can only be cached for
the current vcpu, meaning if two vcpus map the same machine frame number many
times, there will be two valid mappings, usable by every vcpus of the domain.
Most of the time, it is what you want because the operations on machine frames
are often local and this way avoid some unecessary sharing between vcpus.
But sometimes, you want to map a machine frame number into a physical address
intended to be used by several vcpus.
That is the goal of the \texttt{map\_domain\_page\_global()} and
\texttt{unmap\_domain\_page\_global()} functions which do the same thing than
their local sisters but the physical addresses do not go throught the local
cache system, instead their use another global mapping mechanism.

\paragraph{}
Now we have seen the interface provided by the map cache system, let us see the
internals and the limitations of it.
The main structures are described in the \texttt{xen/include/asm-x86/domain.h}
header file and are named \dcache, \vcache and \hashent.
In the first time, we will only see how the local mappings are performed.
The global mapping will be described briefly because it is based on other
mechanisms of xen.

\paragraph{}
The mapping is performed using the linear mapping tables which are tables of
page table entries, belonging to the domain page table\footnote{I have not
actually seen where, but it is very likely to be so}, and mapped on physical
addresses, so their are directly accessible.
That means changes written in these tables are directly seen by the MMU (if
the TLB flushes are performed correctly).
The structure \dcache is the one responsible to maintain the information of
what is mapped where.
It is mainly composed of two bitsets, each bit of these correspond to an entry
in the linear tables.
The first one is \inuse and tracks if a machine frame has been mapped on a
given entry since the last TLB flush.
The second one is \garbage and tracks what entry has been unmapped on the
linear table but not yet flushed.
The goal to have these to bitsets is to batch the TLB flushes since it is a
very costly operation.

\paragraph{}
In addition of these bitsets, come the fields \texttt{entries} and
\texttt{cursor} which indicate respectively the capacity of the bitsets (it is
a constant field), and what is the next entry to search on to find a free slot.
Please notice it does not mean the entrie is actually free, it is just a way
to check the entries in a circular way instead of checking from the entry 0
each time.
The \dcache structure is depicted in figure \ref{figure dcache scheme}.

\begin{figure}
  \centering
  \begin{tikzpicture}[start chain=1 going above, start chain=2 going right,
      start chain=3 going right, start chain=4 going above, node distance=0]
    \tikzstyle{dcache}=[draw,on chain=1, minimum width=2.3cm,
      minimum height=.8cm]
    \tikzstyle{bitset}=[draw,minimum width=.45cm,minimum height=.5cm]
    \tikzstyle{inuse}=[bitset,on chain=2]
    \tikzstyle{garbage}=[bitset,on chain=3]
    \tikzstyle{linear}=[draw,on chain=4,minimum width=1.6cm,
      minimum height=.4cm]
    \def\z{\texttt{0}}
    \def\o{\texttt{1}}

    \node[dcache,anchor=south east] (garbage) at (0,.2) {\garbage};
    \node[dcache] (inuse) {\inuse};
    \node[dcache] (cursor) {\texttt{cursor}};
    \node[dcache] (entries) {\texttt{entries}};
    \node[anchor=north] at ($(garbage.south)+(0,-.3)$) {\dcache};

    \node[garbage,anchor=south west] (garbage0) at (1,0) {\o};
    \foreach \i in {1,2,3} \node[garbage] (garbage\i) {\z};
    \foreach \i in {4,5,6,7} \node[garbage] (garbage\i) {\z};

    \node[inuse,anchor=south west] (inuse0) at (1,1) {\o};
    \foreach \i in {1,2,3} \node[inuse,fill=gray!30] (inuse\i) {\o};
    \node[inuse] (inuse4) {\z};
    \node[inuse] (inuse5) {\z};
    \node[inuse,fill=gray!30] (inuse6) {\o};
    \node[inuse] (inuse7) {\z};

    \node[on chain=4,minimum height=1cm] at (7,-.6)
         {\texttt{\_\_linear\_l1\_table}};
    \node[linear] (linear7) {};
    \node[linear,fill=gray!30] (linear6) {};
    \foreach \i in {5,4} \node[linear] (linear\i) {};
    \foreach \i in {3,2,1} \node[linear,fill=gray!30] (linear\i) {};
    \node[linear] (linear0) {};

    \draw[->] (inuse.east) -- (inuse0.west);
    \draw[->] (garbage.east) -- (garbage0.west);

    \draw[<->] ($(garbage5.south west)+(0,-.3)$) -- node[below] {bit}
    ($(garbage5.south east)+(0,-.3)$);
    \node (entrieswest) at ($(inuse0.north west)+(0,1)$) {};
    \node (entrieseast) at ($(inuse7.north east)+(0,1)$) {};
    \draw[<->] (entrieswest.center) -- node (entriescenter) {}
    (entrieseast.center);
    \draw[dashed,color=gray] (inuse0.north west) -- (entrieswest.center);
    \draw[dashed,color=gray] (inuse7.north east) -- (entrieseast.center);
    \draw[<->] ($(linear0.north west)+(-.2,0)$) --
    node[left] {\texttt{l1\_pgentry\_t}} ($(linear0.south west)+(-.2,0)$);

    \draw[->,rounded corners] (cursor.east) -| (inuse4.north);
    \draw[rounded corners] (entries.east) -| (entriescenter.center);

  \end{tikzpicture}
  \caption{\label{figure dcache scheme}Representation of the \dcache with
    the \inuse and \garbage bitsets.
    Beside the linear mapping \texttt{\_\_linear\_l1\_table} with mapping slots
    in gray.
    These slots match the entries which are \texttt{1} in the \inuse bitset and
    \texttt{0} in the \garbage one.}
\end{figure}

\paragraph{}
When the \map function is called, the map cache system will:
\begin{itemize}
\item try to find a cache entry which already map it (we will see it later)
\item if it cannot find one, it searches, starting from the cursor, an entry in
  the \inuse bitset which is \texttt{0}
\item if it reaches the end of the bitset without finding, it will trigger a
  first ``garbage collection'', meaning all entries which have a \texttt{1}
  in the \garbage bitset will be set to \texttt{0} in both bitsets, then
  it reset the cursor and restart a search
\item if the first collection does not free any entry, a second
  ``garbage collection'' occurs and the map cache system try to evict a cache
  entry (we will see it later) and steal its \inuse entry
\item if the second collection does not find a cache entry which can be evicted
  (because their all have a positive reference counter), Xen will crash
\end{itemize}
If a collection occurs, the map cache system will also perform a TLB flush so
the MMU can know the garbage entries are not mapped anymore.
When the map cache system has an entry, it sets the matching \inuse bit to
\texttt{1} and set the cursor to the next entry.
Then it writes the machine frame number in the appropriate slot of the linear
table and return the physical address, which is simply a predefined physical
address incremented by as many pages as the entry number.

\paragraph{}
When the \unmap function is called, the map cache system starts by computing
the entry number, basing on the given physical address.
With this entry, it can read into the linear table what is the corresponding
machine frame number.
With this last, it can check if there already is a cache entry for this machine
frame number, if so, it just decrement the reference count.
