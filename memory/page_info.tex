\newcommand{\pageinfo}{\texttt{page\_info}\xspace}
\newcommand{\countinfo}{\texttt{count\_info}\xspace}
\newcommand{\ulong}{\texttt{unsigned long}\xspace}
\newcommand{\mfn}{\texttt{mfn_t}\xspace}

\paragraph{}
The \pageinfo structure is the place to find an information about
a machine page frame.
For instance, this is where Xen stores the count of pointer on the page frame
to avoid uncontrolled deallocation.
You can access the \pageinfo of a given page frame address (\ulong) with the
macro \texttt{mfn\_to\_page()} which takes the page frame address as an
argument and brings you back a pointer on the good \pageinfo.
Of course, you can do the reverse operation with the macro
\texttt{page\_to\_mfn()}.

\paragraph{}
You can find the definition of the \pageinfo structure in the header file
\texttt{xen/include/asm-x86/mm.h}.
Because there can be a lot of page frames in a machine and so a lot of
\pageinfo, which have to always remain in memory, the size of this structure
does matter.
That is why, the bits in this structure are carefully used and have different
significations depending on how the page frame is used.

\paragraph{}
The first field is a double word indicating how the page frame is linked to
the other ones.
For most cases, the field to use is \texttt{list} which is the two
``short pointers'' on the previous and on the next \pageinfo into a doubly
linked list.
It can be the free list the page frame belong to, or any page list of a
domain\footnote{not exhaustive}.
The other fields \texttt{up} and \texttt{sharing} are either for shadow paging,
which is not used anymore, and page sharing, which is a corner case.

\paragraph{}
The next field \countinfo is a double word reference counter.
The principle is as usual, each \pageinfo tracks the count of pointer on the
corresponding page frame.
When trying to deallocate a page frame, the reference count is checked, and if
it is not 0, it should not be considered as free.
Now because it is very unlike this counter reach big values, the highter bits
of this counter are used to store other informations.
These informations are described in figure \ref{figure page_info.count_info}
and in table \ref{table page_info.count_info}.
This description stands for a \emph{x86\_64} architecture.
To be portable, you should use the macros listed in
\texttt{xen/include/asm-x86/mm.h} and starting with \texttt{PGC\_}.

\begin{figure}
  \centering
  \newcommand{\bit}[1]{\multicolumn{1}{r}{\scriptsize#1}}
  \newcommand{\field}[1]{\vspace{1pt}\scriptsize#1}
  \begin{tabularx}{\textwidth}
    { | >{\centering}p{10pt}
      | >{\centering}p{10pt}
      | >{\centering}p{10pt}
      | >{\centering}p{10pt}
      | >{\centering}p{10pt}
      | >{\centering}p{10pt}
      | >{\centering}p{10pt}
      | >{\centering}p{30pt}
      | >{\centering\arraybackslash}X | }
    \bit{63} & \bit{62} & \bit{61} & \bit{60} & \bit{59} & \bit{58} &
    \bit{57} & \bit{55} & \bit{0} \\
    \hline
    \field{AL} & \field{XH} & \field{PT} & \field{PA} & \field{PC} &
    \field{PW} & \field{BR} & \field{ST} & \field{CNT}\vspace{8pt} \\
    \hline
  \end{tabularx}
  \caption{\label{figure page_info.count_info}Definition of
    \texttt{page\_info.count\_info}}
\end{figure}

\begin{table}
  \centering
  \begin{tabularx}{\textwidth}{ | l | X | }
    \hline
    field   & description \\
    \hline
    AL      & Allocated bit. If this is set, then the corresponding page frame
              is already allocated to a domain. Obtained with
              \texttt{PGC\_allocated}. \\
    \hline
    XH      & Xen heap bit. If this is set, then the corresponding page frame
              is belonging to the xen heap of the owning domain, meaning it
              can be used for hypervisor operations but not from inside the
              VM\footnote{not sure yet}. Obtained with
              \texttt{PGC\_xen\_heap}. \\
    \hline
    PT      & Page table bit. If this is set, then the corresponding page frame
              is used as the page frame of the owning domain, and so it cannot
              be used from inside the VM. Obtained with
              \texttt{PGC\_page\_table}. \\
    \hline
    PA      & Page attribute table bit. Indicate the state of PAT use for the
              corresponding page frame. Obtained with
              \texttt{PGC\_cacheattr\_mask} and
              \texttt{PGC\_cacheattr\_base}. \\
    \hline
    PC      & Page cache disable bit. Indicate the state of PCD use for the
              corresponding page frame. Obtained with
              \texttt{PGC\_cacheattr\_mask} and
              \texttt{PGC\_cacheattr\_base}. \\
    \hline
    PW      & Page write throught bit. Indicate the state of the PWT use for
              the corresponding page frame. Obtained with
              \texttt{PGC\_cacheattr\_mask} and
              \texttt{PGC\_cacheattr\_base}. \\
    \hline
    BR      & Page broken bit. If this is set, then the page cannot be used
              anymore. Typically, it cannot be returned by a Xen allocation
              function. Obtained with \texttt{PGC\_broken}. \\
    \hline
    ST      & The state of the page. These two bits indicate the state of the
              page frame which can be \texttt{inuse}, \texttt{offlining},
              \texttt{offlined} or \texttt{free}. Obtained with
              \texttt{page\_state\_is()}. \\
    \hline
    CNT     & The reference counter of the page frame. These 55 bits indicate
              how many entities have a reference on the page frame. When the
              count falls to 0, the page can be freed. Obtained with
              \texttt{PGC\_count\_mask}. \\
    \hline
  \end{tabularx}
  \caption{\label{table page_info.count_info}Definition of
    \texttt{page\_info.count\_info} fields}
\end{table}

\paragraph{}
The next field is a double word with a meaning depending on the value of
\countinfo.
If this last is 0, then the related page frame is used as a shadow.
Because it is a corner case, we leave this value.
If \texttt{count\_info \& PGC\_count\_mask} (the mask for the \texttt{counter}
field of \countinfo) is 0, then the page is not used and is in a free list, and
the first 8-bit character of this field is a boolean which is set, means the
TLB has to be flushed before the next use of the page frame
(\texttt{page\_info.u.free.need\_tlbflush}).
If \texttt{count\_info \& PGC\_count\_mask} is not 0, then the page is in use,
and this field is another reference counter
(\texttt{page\_info.u.inuse.type\_info}).

\paragraph{}
This counter is another regular reference counter, but instead indicating the
count of entities which have references to the page frame, it indicates the
count of entities using the page frame with a specific meaning.
For instance, a page frame can be used as a part of a page frame, or as a page
writable by the guest.
This count is keeped to ensure a page cannot change its type while being used.
Once again, because this counter will not reach hight values, the hightest bits
of this counter are used to store valuable informations.
These informations are described in figure \ref{figure page_info.type_info}
and in table \ref{table page_info.type_info}.
This description stands for a \emph{x86\_64} architecture.
To be portable, you should use the macros listed in
\texttt{xen/include/asm-x86/mm.h} and starting with \texttt{PGT\_}.

\begin{figure}
  \centering
  \newcommand{\bit}[1]{\multicolumn{1}{r}{\scriptsize#1}}
  \newcommand{\field}[1]{\vspace{1pt}\scriptsize#1}
  \begin{tabularx}{\textwidth}
    { | >{\centering}p{76pt}
      | >{\centering}p{10pt}
      | >{\centering}p{10pt}
      | >{\centering}p{10pt}
      | >{\centering}p{10pt}
      | >{\centering}p{10pt}
      | >{\centering\arraybackslash}X | }
    \bit{60} & \bit{59} & \bit{58} & \bit{57} & \bit{56} & \bit{55} &
    \bit{0} \\
    \hline
    \field{TYPE} & \field{PI} & \field{VA} & \field{AE} & \field{PA} &
    \field{LK} & \field{CNT}\vspace{8pt} \\
    \hline
  \end{tabularx}
  \caption{\label{figure page_info.type_info}Definition of
    \texttt{page\_info.u.inuse.type\_info}}
\end{figure}

\begin{table}
  \centering
  \begin{tabularx}{\textwidth}{ | l | X | }
    \hline
    field   & description \\
    \hline
    TYPE    & The type of the page. The possible types are listed in
              \ref{table page_info.type_info list}.
              Obtained with \texttt{PGT\_type\_mask}. \\
    \hline
    PI      & Page pinned bit. If this is set, then the owning domain has
              pinned the corresponding page frame to its current type.
              Obtained with \texttt{PGT\_pinned}. \\
    \hline
    VA      & Page validated bit. If this is set, then the corresponding
              page frame has been validated to be of its current type.
              Obtained with \texttt{PGT\_validated}. \\
    \hline
    AE      & PAE xen private bit. If this is set, then the corresponding
              page is a L2 page (second level of page table) and contains
              mapping the guest cannot use. This can be used in PAE only.
              Obtained with \texttt{PGT\_pae\_xen\_l2}. \\
    \hline
    PA      & Partial validated bit. If this is set, then the corresponding
              page frame has been partially validated to be of its current
              type. Obtained with \texttt{PGT\_partial}. \\
    \hline
    LK      & Locked page bit. If this is set, then the corresponding page
              frame is locked. Obtained with \texttt{PGT\_locked}. \\
    \hline
    CNT     & The reference counter of the page frame. These 55 bits indicate
              how many entities have a reference on the page frame as a typed
              page frame. When the count falls to 0, the page can change its
              type. Obtained with \texttt{PGT\_count\_mask}. \\
    \hline
  \end{tabularx}
  \caption{\label{table page_info.type_info}Definition of
    \texttt{page\_info.u.type\_info} fields}
\end{table}

\begin{table}
  \centering
  \begin{tabularx}{\textwidth}{ | l | X | }
    \hline
    type                          & description \\
    \hline
    \texttt{PGT\_none}            & This page has no type. \\
    \hline
    \texttt{PGT\_l1\_page\_table} & This page is used as a first level page
                                    in the page table of the guest. \\
    \hline
    \texttt{PGT\_l2\_page\_table} & This page is used as a second level page
                                    in the page table of the guest. \\
    \hline
    \texttt{PGT\_l3\_page\_table} & This page is used as a third level page
                                    in the page table of the guest. \\
    \hline
    \texttt{PGT\_l4\_page\_table} & This page is used as a fourth level page
                                    in the page table of the guest. \\
    \hline
    \texttt{PGT\_seg\_desc\_page} & This page is used a the \texttt{GDT} or
                                    \texttt{LDT} of the guest. \\
    \hline
    \texttt{PGT\_writable\_page}  & This page is used directly by the guest,
                                    and is writable from inside of it. \\
    \hline
    \texttt{PGT\_shared\_page}    & This page is a shared copy-on-write
                                    page. \\
    \hline
  \end{tabularx}
  \caption{\label{table page_info.type_info list}Definition of
    \texttt{page\_info.u.type\_info} types}
\end{table}

\paragraph{}
The next field is a single word with a meaning depending on the value of
\countinfo.
Again, because the shadow paging is a corner case, we will not describe it
here.
If the page is in a free list, the field indicates the order of the page:
the consecutive pages which are in a free list are packed together to form
power of two sized page frames (\texttt{page\_info.v.free.order}).
At allocation time, their can be taken as a big page frame, or be splitted to
be used separatedely.
The value 0 of this field means a 4 Kilobytes page frame, then it increases in
power of two.
If the page is in use, the field indicates what domain is owning the page
(\texttt{page\_info.v.inuse.\_domain}).
This field should not be used directly, instead, user should apply the macro
\texttt{page\_get\_owner()} of \texttt{page\_set\_owner()} on the \pageinfo.

\paragraph{}
The last field is a single word with a meaning depending on several other
fields.
If the page is free or typeless, then it is a timestamp for the last TLB flush
of this page frame (\texttt{page\_info.tlbflush\_timestamp}).
If the page is partially validated, this field how many \texttt{PTE} has been
validated.
We will not going deeper in the process of page type validation.
